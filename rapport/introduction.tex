\chapter{Introduction}
\label{chap:intro}

La qualité d'un vin est un paramètre qui varie en fonction du goût de tout un chacun. Est-il possible de déterminer la qualité
d'un vin de façon plus objective ? Pour cela, nous pouvons utiliser les caractéristiques physiques et chimiques du vin.
Tout d'abords, nous devons déterminer la nature du problème sur lequel nous allons travailler, s'il s'agit d'un problème de
classification, de régression ou de clustering.

Après avoir déterminé qu'il s'agissait à la fois d'un problème de classification et de régression, nous avons mis en place des
réseaux de neurones adaptés à chacun de ces problèmes. Nous avons évalué les performances de ces réseaux sur les données de vins
rouges et de vins blancs. Nous avons également évalué les performances en utilisant un nombre réduit de caractéristiques. Et enfin,
nous allons conclure ce rapport en présentant ce que nous avons pu tirer de ces expérimentations.